%!TEX root = ../report.tex
\chapter{Implementation}

\textit{[Peer-to-peer] is a specific form of relational dynamic, is based on the assumed equipotency of its participants, organized through the free cooperation of equals in view of the performance of a common task, for the creation of a common good, with forms of decision-making and autonomy that are widely distributed throughout the network.} -- Michel Bauwens~\cite{p2pfoundation:definition}

%Components
%	Python bindings
%	GUI
	
%\section{McChat operations in terms of Novinet primitives}

%TBD: Key algorithms with examples of message exchanges

\section{Note}

The initial intent for this section was to describe the implementation of the application as well as all the relevant knowledge required to deploy the application in a different setup. However, during the initial testing phase of the libraries and associated tools, many deployment issues were met. Addressing them and trying to figure out the remaining unresolved issues took all the time that was available for the project. The issues and how they were addressed is explained in the next section.

A further discussion, is therefore differed until deployment issues are resolved, which means that it won't happen in the context of the course COMP-512 course.

%\section{Bootstrapping}

%\subsection{Network}

%TBD: Command line operations for bringing up the initial nodes.

%\subsection{Session}

%TBD: Procedure for connecting to the network and send messages.

